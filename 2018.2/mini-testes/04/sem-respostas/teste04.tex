\documentclass[12pt,a4paper,oneside]{article}

\usepackage[utf8]{inputenc}
\usepackage[portuguese]{babel}
\usepackage[T1]{fontenc}
\usepackage{amsmath}
\usepackage{amsfonts}
\usepackage{amssymb}
\usepackage{graphicx}
\usepackage{xcolor}
\usepackage{multicol}
% Definindo novas cores
\definecolor{verde}{rgb}{0.25,0.5,0.35}

\author{\\Universidade Federal de Jataí (UFJ)\\Bacharelado em Ciência da Computação \\Linguagens Formais e Autômatos \\Esdras Lins Bispo Jr.}

\date{22 de novembro de 2018}

\title{\sc \huge Mini-Teste 4}

\begin{document}

\maketitle

{\bf ORIENTAÇÕES PARA A RESOLUÇÃO}

\small
 
\begin{itemize}
	\item A avaliação é individual, sem consulta;
	\item A pontuação máxima desta avaliação é 10,0 (dez) pontos, sendo uma das 06 (seis) componentes que formarão a média final da disciplina: quatro mini-testes (MT), uma prova final (PF), exercícios-bônus (EB) e exercícios aplicados em sala de aula pelo método de Instrução pelos Colegas (IpC);
	\item A média final ($MF$) será calculada assim como se segue
	\begin{eqnarray}
		MF & = & MIN(10, S) \nonumber \\
		S & = & [(\sum_{i=1}^{4} max(MT_i, SMT_i ) + PF].0,2  + EB + IpC\nonumber
	\end{eqnarray}
	em que 
	\begin{itemize}
		\item $S$ é o somatório da pontuação de todas as avaliações, e
		\item $SMT_i$ é a substitutiva do mini-teste $i$.
	\end{itemize}
	\item O conteúdo exigido desta avaliação compreende o seguinte ponto apresentado no Plano de Ensino da disciplina: (6) Gramáticas Livres-do-Contexto, (7) Autômatos com Pilha, e (8) Linguagens Não-Livres-do-Contexto.
\end{itemize}

\begin{center}
	\fbox{\large Nome: \hspace{10cm}}
\end{center}

\newpage

\begin{enumerate}
	
	\section*{Quarto Mini-Teste}
	
	\item (5,0 pt) {\bf [Sipser 2.2 Adaptada]} 
	\begin{enumerate}
		\item (2,5 pt) Use as linguagens $A = \{a^m b^n c^n$ | $m, n \geq 0\}$ e $B = \{a^n b^n c^m$ | $m, n \geq 0\}$ juntamente com o fato de que $C = \{a^n b^n c^n$ | $n \geq 0\}$ não é LLC para mostrar que a classe das LLCs não é fechada sob interseção.
		\item (2,5 pt) Use a parte (a) e a lei de DeMorgan ($\overline{A \cap B} = \overline{A} \cup \overline{B}$) para mostrar que a classe de linguagens livres-do-contexto não é fechada sob complementação.
	\end{enumerate}
	
	\item (5,0 pt) {\bf [Sipser 2.4 / 2.6]}  Dê gramáticas livres-do-contexto que gerem as seguintes linguagens. Em todos os itens o alfabeto $\Sigma$ é $\{0,1\}$.
	\begin{enumerate}
		\item (2,0 pt) $\{\omega$ | $\omega$ é um palíndromo $\}$
		\item (3,0 pt) O complemento da linguagem $\{0^n 1^n$ | $n \geq 0 \}$
	\end{enumerate}

\end{enumerate}

\end{document}