\documentclass[12pt,a4paper,oneside]{article}

\usepackage[utf8]{inputenc}
\usepackage[portuguese]{babel}
\usepackage[T1]{fontenc}
\usepackage{amsmath}
\usepackage{amsfonts}
\usepackage{amssymb}
\usepackage{graphicx}
\usepackage{xcolor}
\usepackage{multicol}
% Definindo novas cores
\definecolor{verde}{rgb}{0.25,0.5,0.35}

\author{\\Universidade Federal de Jataí (UFJ)\\Bacharelado em Ciência da Computação \\Linguagens Formais e Autômatos \\Esdras Lins Bispo Jr.}

\date{11 de outubro de 2018}

\title{\sc \huge Mini-Teste 3}

\begin{document}

\maketitle

{\bf ORIENTAÇÕES PARA A RESOLUÇÃO}

\small
 
\begin{itemize}
	\item A avaliação é individual, sem consulta;
	\item A pontuação máxima desta avaliação é 10,0 (dez) pontos, sendo uma das 06 (seis) componentes que formarão a média final da disciplina: quatro mini-testes (MT), uma prova final (PF), exercícios-bônus (EB) e exercícios aplicados em sala de aula pelo método de Instrução pelos Colegas (IpC);
	\item A média final ($MF$) será calculada assim como se segue
	\begin{eqnarray}
		MF & = & MIN(10, S) \nonumber \\
		S & = & [(\sum_{i=1}^{4} max(MT_i, SMT_i ) + PF].0,2  + EB + IpC\nonumber
	\end{eqnarray}
	em que 
	\begin{itemize}
		\item $S$ é o somatório da pontuação de todas as avaliações, e
		\item $SMT_i$ é a substitutiva do mini-teste $i$.
	\end{itemize}
	\item O conteúdo exigido desta avaliação compreende o seguinte ponto apresentado no Plano de Ensino da disciplina: (3) Autômatos Finitos Não-determinísticos, (4) Expressões Regulares, e (5) Linguagens não-regulares.
\end{itemize}

\begin{center}
	\fbox{\large Nome: \hspace{10cm}}
\end{center}

\newpage

\begin{enumerate}
	
	\section*{Terceiro Mini-Teste}
	
	\item {\bf [Sipser 1.22]} Em algumas linguagens de programação, os comentários aparecem entre delimitadores tais como {\tt /\#} e {\tt \#/} . Seja $C$ a linguagem de todas as cadeias válidas de comentários delimitados. Um membro de $C$ deve começar com {\tt /\#} e terminar com {\tt \#/}. Por questões de simplicidade, diremos que os comentários propriamente ditos serão escritos apenas com os símbolos {\tt a} e {\tt b}. Logo, o alfabeto de $C$ é $\Sigma = \{${\tt a}, {\tt b}, {\tt /}, {\tt \#}$\}$. 
	\begin{enumerate}
		\item (2,5 pt) Dê um AFD que reconhece $C$.
		\item (2,5 pt) Dê uma expressão regular que gera $C$.
	\end{enumerate}
	
	\vspace*{0.5cm}
	
	\item (5,0 pt) Seja a linguagem $A = \{\omega \omega \omega$ | $\omega \in \{a,b\}^*\}$. Mostre o por quê da cadeia $0^p0^{2p}$ não poder ser utilizada para provar que $A$ {\bf não} é regular (em que $p$ é o comprimento do bombeamento).

\end{enumerate}

\end{document}