\documentclass[12pt,a4paper,oneside]{article}

\usepackage[utf8]{inputenc}
\usepackage[portuguese]{babel}
\usepackage[T1]{fontenc}
\usepackage{amsmath}
\usepackage{amsfonts}
\usepackage{amssymb}
\usepackage{graphicx}
\usepackage{xcolor}
\usepackage{multicol}
\usepackage{tikz}
\usetikzlibrary{automata,positioning}

% Definindo novas cores
\definecolor{verde}{rgb}{0.25,0.5,0.35}

\author{\\Universidade Federal de Jataí (UFJ)\\Bacharelado em Ciência da Computação \\Linguagens Formais e Autômatos \\Esdras Lins Bispo Jr.}

\date{23 de agosto de 2018}

\title{\sc \huge Mini-Teste 1}

\begin{document}

\maketitle

{\bf ORIENTAÇÕES PARA A RESOLUÇÃO}

\small
 
\begin{itemize}
	\item A avaliação é individual, sem consulta;
	\item A pontuação máxima desta avaliação é 10,0 (dez) pontos, sendo uma das 06 (seis) componentes que formarão a média final da disciplina: quatro mini-testes (MT), uma prova final (PF), exercícios-bônus (EB) e exercícios aplicados em sala de aula pelo método de Instrução pelos Colegas (IpC);
	\item A média final ($MF$) será calculada assim como se segue
	\begin{eqnarray}
		MF & = & MIN(10, S) \nonumber \\
		S & = & [(\sum_{i=1}^{4} max(MT_i, SMT_i ) + PF].0,2  + EB + IpC\nonumber
	\end{eqnarray}
	em que 
	\begin{itemize}
		\item $S$ é o somatório da pontuação de todas as avaliações, e
		\item $SMT_i$ é a substitutiva do mini-teste $i$.
	\end{itemize}
	\item O conteúdo exigido desta avaliação compreende o seguinte ponto apresentado no Plano de Ensino da disciplina: (1) Revisão de Fundamentos e (2) Autômatos Finitos Determinísticos.
\end{itemize}

\begin{center}
	\fbox{\large Nome: \hspace{10cm}}
\end{center}

\newpage

\begin{enumerate}
	
	\section*{Primeiro Teste}
	
	\item (5,0 pt) {\bf [Sipser 0.10]} Encontre o erro na seguinte prova de que 2=1.
	\begin{flushleft}
		Considere a equação $a=b$. Multiplique ambos os lados por $a$ para obter $a^2 = ab$. Subtraia $b^2$ de ambos os lados para obter $a^2 - b^2 = ab-b^2$. Agora fatore cada lado, obtendo $(a+b)(a-b) = b(a-b)$, e divida cada lado por $(a-b)$, para chegar em $a+b = b$. Finalmente, faça $a$ e $b$ iguais a 1, o que mostra que 2=1.
	\end{flushleft}

	{\color{blue} {\bf Resposta: } O erro na prova está no passo em que se divide ambos os lados por $(a-b)$. Como a equação inicial é $a=b$, logo $a-b = 0$. Assim é impossível realizar uma divisão por zero.
	}
	
	\vspace{0.3cm}
	
	\item (5,0 pt) Dê o diagrama de estados dos AFDS que reconhecem as seguintes linguagens. Admita em todos os itens que o alfabeto é  $\{a,b\}$.
		\begin{enumerate}
			\item {\bf [Sipser 1.4 (a)]} (1,0 pt) \\$\{\omega$ | $\omega$ tem pelo menos três {\sf a}s e pelo menos dois {\sf b}s$\}$
			
			\vspace*{0.3cm}
			
			{\color{blue}
				
				\begin{tikzpicture}[->,>=stealth,shorten >=1pt,auto,node distance=1.5cm,
				semithick]
				\node[state,initial] 	(q11)  						{$q_{1,1}$}; 
				\node[state] 			(q12)	[right=of q11]		{$q_{1,2}$};
				\node[state] 			(q13)	[right=of q12]		{$q_{1,3}$};
				\node[state] 			(q14)	[right=of q13]		{$q_{1,4}$};
				
				\node[state] 			(q21)	[below=of q11]		{$q_{2,1}$};
				\node[state] 			(q22)	[right=of q21]		{$q_{2,2}$};
				\node[state] 			(q23)	[right=of q22]		{$q_{2,3}$};
				\node[state] 			(q24)	[right=of q23]		{$q_{2,4}$};
				
				\node[state] 			(q31)	[below=of q21]		{$q_{3,1}$};
				\node[state] 			(q32)	[right=of q31]		{$q_{3,2}$};
				\node[state] 			(q33)	[right=of q32]		{$q_{3,3}$};
				\node[state, accepting] 			(q34)	[right=of q33]		{$q_{3,4}$};
				
				\path[->] 
				
				% TRANSIÇÕES DE SÍMBOLOS a
				(q11) edge [bend left]	node {a} (q12)
				(q12) edge [bend left]	node {a} (q13)
				(q13) edge [bend left]	node {a} (q14)
				(q14) edge [loop right] 	node {a} ()
				
				(q21) edge [bend left]	node {a} (q22)
				(q22) edge [bend left]	node {a} (q23)
				(q23) edge [bend left]	node {a} (q24)
				(q24) edge [loop right] 	node {a} ()
				
				(q31) edge [bend left]	node {a} (q32)
				(q32) edge [bend left]	node {a} (q33)
				(q33) edge [bend left]	node {a} (q34)
				(q34) edge [loop right] 	node {a,b} ()
				
				% TRANSIÇÕES DE SÍMBOLOS b 
				(q11) edge [bend right]	node {b} (q21)
				(q21) edge [bend right]	node {b} (q31)
				(q31) edge [loop below] 	node {b} ()
				
				(q12) edge [bend right]	node {b} (q22)
				(q22) edge [bend right]	node {b} (q32)
				(q32) edge [loop below] 	node {b} ()
				
				(q13) edge [bend right]	node {b} (q23)
				(q23) edge [bend right]	node {b} (q33)
				(q33) edge [loop below] 	node {b} ()
				
				(q14) edge [bend right]	node {b} (q24)
				(q24) edge [bend right]	node {b} (q34);
				\end{tikzpicture}
			}
		
			\item {\bf [Sipser 1.5 (g)]} (2,0 pt) $\{\omega$ | $\omega$ é qualquer cadeia que não contém exatamente dois {\sf a}s$\}$
			
			\vspace*{0.3cm}
			
			{\color{blue}
				
				\begin{tikzpicture}[->,>=stealth,shorten >=1pt,auto,node distance=1.5cm,
				semithick]
				\node[state,initial,accepting] 	(q1)  						{$q_1$}; 
				\node[state, accepting] 		(q2)	[right=of q1]		{$q_2$};
				\node[state] 					(q3)	[right=of q2]		{$q_3$};
				\node[state, accepting] 		(q4)	[below=of q2]		{$q_4$};
				
				\path[->] 
				
				(q1) edge [bend left]	node {a} (q2)
				(q1) edge [loop above] 	node {b} ()
				
				(q2) edge [bend left]	node {a} (q3)
				(q2) edge [loop above] 	node {b} ()
				
				(q3) edge [bend left]	node {a} (q4)
				(q3) edge [loop above] 	node {b} ()
				
				(q4) edge [loop above] 	node {a,b} ();
				
				\end{tikzpicture}
			}
		
			\item {\bf [Sipser 1.6 (h)]} (2,0 pt) $\{\omega$ | {\color{blue} $\omega$ contém} qualquer subcadeia exceto {\color{blue} {\sf aa}} e {\color{blue} {\sf aaa}}$\}$ 
			
			\vspace*{0.3cm}
			
			{\color{blue}
				
				\begin{tikzpicture}[->,>=stealth,shorten >=1pt,auto,node distance=1.5cm,
				semithick]
				\node[state,initial,accepting] 	(q1)  						{$q_1$}; 
				\node[state, accepting] 		(q2)	[right=of q1]		{$q_2$};
				\node[state] 					(q3)	[right=of q2]		{$q_3$};
				
				\path[->] 
				
				(q1) edge [bend left]	node {a} (q2)
				(q1) edge [loop above] 	node {b} ()
				
				(q2) edge [bend left]	node {a} (q3)
				(q2) edge [bend left]	node {b} (q1)
				
				(q3) edge [loop above] 	node {a,b} ();
				
				\end{tikzpicture}
			}
		\end{enumerate}

\end{enumerate}

\end{document}