\documentclass[12pt,a4paper,oneside]{article}

\usepackage[utf8]{inputenc}
\usepackage[brazilian]{babel}
\usepackage[T1]{fontenc}
\usepackage{amsmath}
\usepackage{amsfonts}
\usepackage{amssymb}
\usepackage{graphicx}
\usepackage{xcolor}
\usepackage{multicol}
% Definindo novas cores
\definecolor{verde}{rgb}{0.25,0.5,0.35}

\author{\\Universidade Federal de Jataí (UFJ)\\Bacharelado em Ciência da Computação \\Linguagens Formais e Autômatos \\Esdras Lins Bispo Jr.}

\date{10 de outubro de 2019}

\title{\sc \huge Mini-Teste 3}

\begin{document}

\maketitle

{\bf ORIENTAÇÕES PARA A RESOLUÇÃO}

\small
 
\begin{itemize}
	\item A avaliação é individual, sem consulta;
	\item A pontuação máxima desta avaliação é 10,0 (dez) pontos, sendo uma das 06 (seis) componentes que formarão a média final da disciplina: quatro mini-testes (MT), uma prova final (PF) e exercícios aplicados em sala de aula pelo método de Instrução pelos Colegas (IpC);
	\item A média final ($MF$) será calculada assim como se segue
	\begin{eqnarray}
		MF & = & MIN(10, S) \nonumber \\
		S & = & [(\sum_{i=1}^{4} max(MT_i, SMT_i ) + PF].0,2  + IpC\nonumber
	\end{eqnarray}
	em que 
	\begin{itemize}
		\item $S$ é o somatório da pontuação de todas as avaliações, e
		\item $SMT_i$ é a substitutiva do mini-teste $i$.
	\end{itemize}
	\item O conteúdo exigido desta avaliação compreende o seguinte ponto apresentado no Plano de Ensino da disciplina: (2) Autômatos Finitos Determinísticos, e (3) Autômatos Finitos Não-determinísticos, (4) Expressões Regulares e (5) Linguagens Não-Regulares.
\end{itemize}

\begin{center}
	\fbox{\large Nome: \hspace{10cm}}
\end{center}

\newpage

\begin{enumerate}
	
	\section*{Terceiro Teste}
	
	\item (5,0 pt) Utilizando expressão regular, mostre que a classe de linguagens regulares é fechada sobre a operação de concatenação.
	
	\vspace*{0.3cm}
	
	{\color{blue}
		{\bf Prova:} Sejam $A$ e $B$ duas linguagens regulares quaisquer. Como $A$ e $B$ são regulares, então existem as expressões regulares (ERs) $R_A$ e $R_B$ que a geram, respectivamente. Pela definição indutiva de ER, se $R_A$ e $R_B$ são ERs, então $R_A \circ R_B$ é uma ER. Como toda ER gera uma linguagem regular, $R_A \circ R_B$ é regular. Logo, a classe de linguagens regulares é fechada sob a operação de concatenação $\blacksquare$
	}
	
	\item (5,0 pt) Aponte o erro (e justifique o motivo) no seguinte argumento que tenta afirmar que $0^*1^*$ não é regular:
	\begin{quote}
		A prova é por contradição. Suponha que $0^*1^*$ seja regular. Seja $p$ o comprimento do bombeamento para $0^*1^*$ dado pelo lema do bombeamento. Escolha para $s$ a cadeia $0^p1^p$. Você sabe que $s$ é um membro de $0^*1^*$, mas $s$ não pode ser bombeada. Assim, temos uma contradição. Logo, $0^*1^*$ não é regular.
	\end{quote}
	Vale lembrar que $0^*1^*$ é regular.
	
		\vspace*{0.3cm}
	
	{\color{blue} 
		{\bf Resposta:} O erro está em afirmar que $s$ não pode ser bombeada. Existem várias formas de dividir $s = xyz$ para satisfazer ao lema. Uma delas é admitir $x = \epsilon$, $y = 0^p$ e $z = 1^p$. Desta forma, é possível bombear $xy^iz$ tanto para cima ($i>0$) quanto para baixo ($i=0$).
	}
	
\end{enumerate}

\end{document}