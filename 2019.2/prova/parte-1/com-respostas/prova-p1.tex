\documentclass[12pt,a4paper,oneside]{article}

\usepackage[utf8]{inputenc}
\usepackage[portuguese]{babel}
\usepackage[T1]{fontenc}
\usepackage{amsmath}
\usepackage{amsfonts}
\usepackage{amssymb}
\usepackage{graphicx}
\usepackage{xcolor}
\usepackage{multicol}
\usepackage{tikz}
\usetikzlibrary{automata,positioning}
% Definindo novas cores
\definecolor{verde}{rgb}{0.25,0.5,0.35}

\author{\\Universidade Federal de Jataí (UFJ)\\Bacharelado em Ciência da Computação \\Linguagens Formais e Autômatos \\Esdras Lins Bispo Jr.}

\date{27 de novembro de 2019}

\title{\sc \huge Prova (Parte 1)}

\begin{document}

\maketitle

{\bf ORIENTAÇÕES PARA A RESOLUÇÃO}

\small
 
\begin{itemize}
	\item A avaliação é individual, sem consulta;
	\item A pontuação máxima desta avaliação é 10,0 (dez) pontos, sendo uma das 06 (seis) componentes que formarão a média final da disciplina: quatro mini-testes (MT), uma prova final (PF), exercícios-bônus (EB) e exercícios aplicados em sala de aula pelo método de Instrução pelos Colegas (IpC);
	\item A média final ($MF$) será calculada assim como se segue
	\begin{eqnarray}
		MF & = & MIN(10, S) \nonumber \\
		S & = & [(\sum_{i=1}^{4} max(MT_i, SMT_i ) + PF].0,2  + EB + IpC\nonumber
	\end{eqnarray}
	em que 
	\begin{itemize}
		\item $S$ é o somatório da pontuação de todas as avaliações, e
		\item $SMT_i$ é a substitutiva do mini-teste $i$.
	\end{itemize}
	\item O conteúdo exigido desta avaliação compreende o seguinte ponto apresentado no Plano de Ensino da disciplina: (1) Revisão de Fundamento, (2) Autômatos Finitos Determinísticos, (3) Autômatos Finitos Não-determinísticos.
\end{itemize}

\begin{center}
	\fbox{\large Nome: \hspace{10cm}}
\end{center}

\newpage

\begin{enumerate}
	
	\section*{Mini-Teste 1}
	
	\item (5,0 pt) {\bf [Sipser 0.6 Adaptada] } Seja $X$ o conjunto $\{1,2,3,4,5\}$ e $Y$ o conjunto $\{6,7,8,9,10\}$. A função unária
	$f : X \rightarrow Y$ e a função binária $g: X \times Y \rightarrow Y$ são descritas nas tabelas seguintes.
	
	\begin{multicols}{2}
		
		\begin{tabular}{c|c}
			$n$	&	$f(n)$	\\
			\hline
			1 	&	6	\\			
			2 	&	7	\\
			3 	&	6	\\
			4 	&	7	\\
			5 	&	6	\\
			\hline
		\end{tabular}
		
		\columnbreak
		
		\begin{tabular}{c|ccccc}
			$g$	& 	6	& 	7	& 	8	& 	9	&	10 \\
			\hline
			1	&	10	&	10	&	10	&	10	&	10 \\
			2	& 	7	& 	8	& 	9	& 	10	& 	6 \\	
			3	& 	7	& 	7	& 	8	& 	8	& 	9 \\
			4	& 	9	& 	8	& 	7	& 	6	& 	10 \\
			5	& 	6	& 	6	& 	6	& 	6	& 	6 \\
			\hline
		\end{tabular}
		
	\end{multicols}
	
	\begin{enumerate}
		
		\item (2,5 pt) Quais são o contradomínio e o domínio de $f$?\\
		{\color{blue} {\bf R:} $Y$ e $X$, nesta ordem.}
		
		\item (2,5 pt) Qual é o valor de $g(4,f(4))$?\\
		{\color{blue} {\bf R:} Como $f(4) =7$, temos que $g(4,f(4)) = g(4,7) = 8$.}
		
	\end{enumerate}

	\item (5,0 pt) {\bf [Sipser 0.5]} Se $C$ é um conjunto com $n$ elementos, quantos elementos estão no conjunto das partes de $C$? Explique sua resposta.

		\vspace*{0.1cm}
	
	{\color{blue} R - O conjunto das partes de $C$ têm $2^n$ elementos. O conjunto das partes de $C$ tem todos os subconjuntos de $C$. Logo, todos os subconjuntos de 0 elemento, de 1 elemento, até todos os subconjuntos de $n$ elementos são membros de $\mathcal{P}(C)$. Assim, $|\mathcal{P}(C)| = C^n_0 + C^n_1 \ldots C^n_n = 2^n$.
	}

\newpage

	\section*{Mini-Teste 2}

\item (5,0 pt) Dê o diagrama de estados das máquinas que reconhecem as seguintes linguagens. Admita em todos os itens que o alfabeto é  $\{0,1\}$.
\begin{enumerate}
	\item {\bf [Sipser 1.6 (i)]} (2,5 pt) |Construir um AFD| \\$\{\omega$ | toda posição ímpar de $\omega$ é um 1$\}$
	
	\vspace*{0.3cm}
	
	{\color{blue}
		
		\begin{tikzpicture}[->,>=stealth,shorten >=1pt,auto,node distance=1.5cm,
		semithick]
		\node[state,initial,accepting] 	(q1)  						{$q_1$}; 
		\node[state, accepting] 		(q2)	[right=of q1]		{$q_2$};
		\node[state] 					(q3)	[below=of q2]		{$q_3$};
		
		\path[->] 
		
		(q1) edge [bend left]	node {1} (q2)
		(q1) edge [bend right]	node {0} (q3)
		
		(q2) edge [bend left]	node {0,1} (q1)
		
		(q3) edge [loop below] 	node {0,1} ();
		
		\end{tikzpicture}
	}
	
	\item {\bf [Sipser 1.7 (e)]} (2,5 pt) |Construir um AFN| 
	\\a linguagem $0^*1^*0^+$ com três estados.
	
	\vspace*{0.3cm}
	
	{\color{blue}
		
		\begin{tikzpicture}[->,>=stealth,shorten >=1pt,auto,node distance=1.5cm,
		semithick]
		\node[state,initial] 	(q1)  						{$q_1$}; 
		\node[state] 		(q2)	[right=of q1]		{$q_2$};
		\node[state,accepting] 					(q3)	[right=of q2]		{$q_3$};
		
		\path[->] 
		
		(q1) edge [loop above] 	node {0} ()
		(q1) edge [bend left]	node {$\epsilon$} (q2)
		
		(q2) edge [loop above] 	node {1} ()
		(q2) edge [bend left]	node {0} (q3)
		
		(q3) edge [loop above] 	node {0} ();
		
		\end{tikzpicture}
	}

\end{enumerate}

\newpage

\item (5,0 pt) {\bf [IpC]}
Um autômato finito é definido por uma 5-upla $(Q, \Sigma, \delta, q_0, F)$. A função $\delta$ é definida como se segue
\begin{center}
	$\delta:Q \times \Sigma \rightarrow Q$
\end{center}
Em relação à $\delta$, marque a alternativa \underline{correta} e \underline{justifique} o motivo das demais serem falsas.

\begin{enumerate}
	\item os estados do autômato são necessários apenas no domínio da função.\vspace*{0.3cm}
	
	{ \color{red} {\bf Resposta:} Não é correto. Não apenas no domínio na função, mas no contradomínio também.}
	
	\item o contradomínio da função é o alfabeto.
	
	\vspace*{0.3cm}
	
	{ \color{red} {\bf Resposta:} Não é correto. O contradomínio da função é o conjunto de estados.}
	
	\item as possibilidades de valores de entradas são infinitas.
	
	\vspace*{0.3cm}
	
	{ \color{red} {\bf Resposta:} Não é correto. $Q$ e $\Sigma$ são finitos. Logo, $Q \times \Sigma$ é finito.}
	
	\item é uma função que recebe duas entradas, sendo um estado e um símbolo do alfabeto.
	
	\vspace*{0.3cm}
	
	{ \color{blue} {\bf Resposta:} Correto.}
\end{enumerate}


\end{enumerate}

\end{document}