 \documentclass[xcolor=dvipsnames,table]{beamer}
%o
%e

\usepackage{latexsym}
\usepackage [ansinew]{inputenc}
\usepackage[brazil]{babel}
\usepackage{amssymb} %Este e o AMS paquete
\usepackage{amsmath}
\usepackage{stmaryrd}
\usepackage{fancybox}
\usepackage{datetime}
\usepackage[inline]{enumitem}


\usepackage[T1]{fontenc}

%\usepackage{beamerthemesplit}
\usepackage{graphicx}
\usepackage{graphics}
\usepackage{url}
\usepackage{algorithmic}
\usepackage{algorithm}
\usepackage{acronym}
\usepackage{array}

\newtheorem{definicao}{Definio}
\newcommand{\tab}{\hspace*{2em}}

\mode<presentation>
{
  %\definecolor{colortexto}{RGB}{153,100,0}
  \definecolor{colortexto}{RGB}{0,0,0}
  
% \setbeamersize{sidebar width left=0.5cm}
  \setbeamertemplate{background canvas}[vertical shading][ bottom=white!10,top=white!10]
%   \setbeamercolor{title}{fg=colortitulo!80!black,bg=blue!20!white}
%   \setbeamercolor{title}{bg=colortitulo!20!black}
%   \setbeamercolor{background canvas}{bg=colortitulo}
%   \setbeamercolor{frametitle}{fg=red}

  \setbeamercolor{normal text}{fg=colortexto} 

  \usetheme{Warsaw}
  %\logo{\includegraphics[width=2cm]{Images/ratonfuerte.jpg}}


%   \usefonttheme[onlysmall]{structurebold}
%   \usecolortheme{seahorse}
%  \usecolortheme[named={YellowOrange}]{structure}
%   \usecolortheme[named={Blue}]{structure}
%   \usecolortheme{crane}
%   \useoutertheme{default}
}

\title{Equival�ncia de AFNs e AFDs} 

\author{
  Esdras Lins Bispo Jr. \\ \url{bispojr@ufg.br}
}
 \institute{
  Linguagens Formais e Aut�matos \\Bacharelado em Ci�ncia da Computa��o}
\date{\textbf{18 de setembro de 2019} }

\logo{\includegraphics[width=1cm]{images/ufjLogo.png}}

\begin{document}

	\begin{frame}
		\titlepage
	\end{frame}

	\AtBeginSection{
		\begin{frame}{Sum�rio}%[allowframebreaks]{Sum�rio}
    		\tableofcontents[currentsection]
    		%\tableofcontents[currentsection, hideothersubsections]
		\end{frame}
	}

	\begin{frame}{Plano de Aula}
		\tableofcontents
		%\tableofcontents[hideallsubsections]
	\end{frame}	
	
	\section{Instru��o pelos Colegas}
	
	\begin{frame}{Quest�o 038}	
		\begin{block}{[Q038]}
			� verdade que todo AFN tem um AFD equivalente. Na prova apresentada pelo Sipser, ele constroi um AFD $M$ a partir de um AFN $N$. Se $N$ tem 10 estados, quantos estados teria $M$?
		\end{block}
		\begin{enumerate}[label=(\Alph*)]
			\item 10
			\item 100
			\item $2^{10}$
			\item $10^2$
		\end{enumerate}
	\end{frame}
	
	\begin{frame}{Quest�o 039}	
		\begin{block}{[Q039]}
			� verdade que todo AFN tem um AFD equivalente. Na prova apresentada pelo Sipser, ele constroi um AFD $M = (Q', \Sigma, \delta', q_0', F')$ a partir de um AFN $N = (Q, \Sigma, \delta, q_0, F)$. 
			
			\vspace*{0.3cm}
						
			$Q' = \mathcal{P}(Q)$ porque...
		\end{block}
		\begin{enumerate}[label=(\Alph*)]
			\item sempre um AFD tem mais estados que um AFN.
			\item $\mathcal{P}(Q)$ � o contradom�nio de $\delta$.
			\item $\mathcal{P}(Q)$ � o conjunto de estados de $N$.
			\item o conjunto vazio � subconjunto de qualquer conjunto.
		\end{enumerate}
	\end{frame}

	\begin{frame}{Quest�o 040}	
		\begin{block}{[Q040]}
			� verdade que todo AFN tem um AFD equivalente. Na prova apresentada pelo Sipser, ele constroi um AFD $M = (Q', \Sigma, \delta', q_0', F')$ a partir de um AFN $N = (Q, \Sigma, \delta, q_0, F)$. 
			
			\vspace*{0.3cm}
						
			$F' = \{R \in Q'$ | $R$ cont�m um estado de aceita��o de $N\}$ porque...
		\end{block}
		\begin{enumerate}[label=(\Alph*)]
			\item � poss�vel que $Q' = Q$, ent�o � necess�rio explicitar os estados finais.
			\item se $R \in F'$ ent�o todos os estados que est�o em $R$ s�o finais.
			\item $R$ representa o n�vel da �rvore de execu��o de N em que pelo menos um dos estados � final.
			\item $R$ necessita ser um estado de $Q$, e n�o um conjunto de estados de $Q$.
		\end{enumerate}
	\end{frame}
	
	\begin{frame}
		\titlepage
	\end{frame}
	
\end{document}