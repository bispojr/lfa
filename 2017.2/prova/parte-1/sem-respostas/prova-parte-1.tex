\documentclass[12pt,a4paper,oneside]{article}

\usepackage[utf8]{inputenc}
\usepackage[portuguese]{babel}
\usepackage[T1]{fontenc}
\usepackage{amsmath}
\usepackage{amsfonts}
\usepackage{amssymb}
\usepackage{graphicx}
\usepackage{xcolor}
\usepackage{multicol}
% Definindo novas cores
\definecolor{verde}{rgb}{0.25,0.5,0.35}

\author{\\Universidade Federal de Goiás (UFG) - Regional  Jataí\\Bacharelado em Ciência da Computação \\Linguagens Formais e Autômatos \\Esdras Lins Bispo Jr.}

\date{20 de fevereiro de 2018}

\title{\sc \huge Prova (Parte 1)}

\begin{document}

\maketitle

{\bf ORIENTAÇÕES PARA A RESOLUÇÃO}

\small
 
\begin{itemize}
	\item A avaliação é individual, sem consulta;
	\item A pontuação máxima desta avaliação é 10,0 (dez) pontos, sendo uma das 06 (seis) componentes que formarão a média final da disciplina: quatro testes, uma prova e exercícios-bônus;
	\item A média final ($MF$) será calculada assim como se segue
	\begin{eqnarray}
		MF & = & MIN(10, S) \nonumber \\
		S & = & (\sum_{i=1}^{4} 0,2.T_i ) + 0,2.P  + EB\nonumber
	\end{eqnarray}
	em que 
	\begin{itemize}
		\item $S$ é o somatório da pontuação de todas as avaliações,
		\item $T_i$ é a pontuação obtida no teste $i$,
		\item $P$ é a pontuação obtida na prova, e
		\item $EB$ é a pontuação total dos exercícios-bônus.
	\end{itemize}
	\item O conteúdo exigido desta avaliação compreende o seguinte ponto apresentado no Plano de Ensino da disciplina: (1) Revisão de Fundamentos, (2) Autômatos Finitos Determinísticos, e (3) Autômatos Finitos Não-Determinísticos.
\end{itemize}

\begin{center}
	\fbox{\large Nome: \hspace{10cm}}
\end{center}

\newpage

\begin{enumerate}
	
	\section*{Primeiro Teste}
	
	\item (5,0 pt) {\bf [Sipser 0.5]} Se $C$ é um conjunto com $n$ elementos, quantos elementos estão no conjunto das partes de $C$? Explique sua resposta.
	
	\item (5,0 pt) {\bf [Sipser 1.3]} A descrição formal de um AFD $M$ é \\
	$(\{q_1, q_2, q_3, q_4, q_5\}, \{u,d\}, \delta, q_3, \{q_3\})$ em que $\delta$ é dada pela tabela a seguir.
	
	\begin{center}
		\begin{tabular}{c|cc}
					&	$u$		&	$d$	\\
			\hline
			$q_1$	&	$q_1$ 	&	$q_2$	\\
			$q_2$	&	$q_1$ 	&	$q_3$	\\
			$q_3$	&	$q_2$ 	&	$q_4$	\\
			$q_4$	&	$q_3$ 	&	$q_5$	\\
			$q_5$	&	$q_4$ 	&	$q_5$	\\
			\hline
		\end{tabular}
	\end{center}
	
	Dê o diagrama de estados desta máquina.
	
	\section*{Segundo Teste}
	
	\item (5,0 pt) {\bf [Sipser 1.11]}  Prove que todo AFN pode ser convertido em um AFN equivalente que tenha apenas um único estado final.
	
	\item (5,0 pt) {\bf [Sipser 1.14 (b)]} Mostre através de um exemplo que, se $M$ é um AFN que reconhece a linguagem $C$, trocar os seus estados simples pelos finais (e vice-versa) não garante necessariamente que o novo AFN reconhece o complemento de $C$. A classe de linguagens reconhecidas por AFNs é fechada sob a operação de complemento? Justifique as suas respostas.

\end{enumerate}

\end{document}