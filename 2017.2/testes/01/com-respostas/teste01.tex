\documentclass[12pt,a4paper,oneside]{article}

\usepackage[utf8]{inputenc}
\usepackage[portuguese]{babel}
\usepackage[T1]{fontenc}
\usepackage{amsmath}
\usepackage{amsfonts}
\usepackage{amssymb}
\usepackage{graphicx}
\usepackage{xcolor}
\usepackage{multicol}
\usepackage{tikz}
\usetikzlibrary{automata,positioning}
% Definindo novas cores
\definecolor{verde}{rgb}{0.25,0.5,0.35}

\author{\\Universidade Federal de Goiás (UFG) - Regional  Jataí\\Bacharelado em Ciência da Computação \\Linguagens Formais e Autômatos \\Esdras Lins Bispo Jr.}

\date{13 de novembro de 2017}

\title{\sc \huge Primeiro Teste}

\begin{document}

\maketitle

{\bf ORIENTAÇÕES PARA A RESOLUÇÃO}

\small
 
\begin{itemize}
	\item A avaliação é individual, sem consulta;
	\item A pontuação máxima desta avaliação é 10,0 (dez) pontos, sendo uma das 06 (seis) componentes que formarão a média final da disciplina: quatro testes, uma prova e exercícios-bônus;
	\item A média final ($MF$) será calculada assim como se segue
	\begin{eqnarray}
		MF & = & MIN(10, S) \nonumber \\
		S & = & (\sum_{i=1}^{4} 0,2.T_i ) + 0,2.P  + EB\nonumber
	\end{eqnarray}
	em que 
	\begin{itemize}
		\item $S$ é o somatório da pontuação de todas as avaliações,
		\item $T_i$ é a pontuação obtida no teste $i$,
		\item $P$ é a pontuação obtida na prova, e
		\item $EB$ é a pontuação total dos exercícios-bônus.
	\end{itemize}
	\item O conteúdo exigido desta avaliação compreende o seguinte ponto apresentado no Plano de Ensino da disciplina: (1) Revisão de Fundamentos e (2) Autômatos Finitos Determinísticos.
\end{itemize}

\begin{center}
	\fbox{\large Nome: \hspace{10cm}}
\end{center}

\newpage

\begin{enumerate}
	
	\section*{Primeiro Teste}
	
	\item (5,0 pt) {\bf [Sipser 0.6]} Seja $X$ o conjunto $\{1,2,3,4,5\}$ e $Y$ o conjunto $\{6,7,8,9,10\}$. A função unária
	$f : X \rightarrow Y$ e a função binária $g: X \times Y \rightarrow Y$ são descritas nas tabelas seguintes.
	
	\begin{multicols}{2}
		
		\begin{tabular}{c|c}
			$n$	&	$f(n)$	\\
			\hline
			1 	&	6	\\			
			2 	&	7	\\
			3 	&	6	\\
			4 	&	7	\\
			5 	&	6	\\
			\hline
		\end{tabular}
	
		\columnbreak
		
		\begin{tabular}{c|ccccc}
			$g$	& 	6	& 	7	& 	8	& 	9	&	10 \\
			\hline
			1	&	10	&	10	&	10	&	10	&	10 \\
			2	& 	7	& 	8	& 	9	& 	10	& 	6 \\	
			3	& 	7	& 	7	& 	8	& 	8	& 	9 \\
			4	& 	9	& 	8	& 	7	& 	6	& 	10 \\
			5	& 	6	& 	6	& 	6	& 	6	& 	6 \\
			\hline
		\end{tabular}
	
	\end{multicols}
	
	\begin{enumerate}
		\item (1,0 pt) Qual é o valor de $f(2)$? \\
						{\color{blue} {\bf R:} 7.}
		\item (1,0 pt) Quais são o contradomínio e o domínio de $f$?\\
						{\color{blue} {\bf R:} $Y$ e $X$, nesta ordem.}
		\item (1,0 pt) Qual é o valor de $g(2,10)$?\\
						{\color{blue} {\bf R:} 6.}
		\item (1,0 pt) Quais são o contradomínio e o domínio de $g$?\\
						{\color{blue} {\bf R:} $Y$ e $X \times Y$, nesta ordem.}
		\item (1,0 pt) Qual é o valor de $g(4,f(4))$?\\
						{\color{blue} {\bf R:} Como $f(4) =7$, temos que $g(4,f(4)) = g(4,7) = 8$.}
	\end{enumerate}
	
	\newpage
	
	\item (5,0 pt) Dê o diagrama de estados dos AFDs que reconhecem as seguintes linguagens. Admita em todos os itens que o alfabeto é  $\{a,b\}$.
		\begin{enumerate}
			\item {\bf [Sipser 1.4 (e)]} (2,0 pt) \\$\{\omega$ | $\omega$ começa com um {\sf a} e tem no máximo um {\sf b}$\}$
			
			\vspace*{0.3cm}
			
			{\color{blue}
			
		\begin{tikzpicture}
		[shorten >=1pt,node distance=2cm,on grid,auto] 
			\node[state,initial] (q_0)   {$q_0$}; 
			\node[state, accepting] (q_1) [right=of q_0] {$q_1$}; 
			\node[state, accepting] (q_2) [right=of q_1] {$q_2$}; 
			\node[state](q_3) [below=of q_1] {$q_3$};
			\path[->] 
			(q_0) 	edge  node {a} (q_1)
			 		edge  node  {b} (q_3)
			(q_1)	edge [loop above] node {a} ()
					edge  node  {b} (q_2)
			(q_2) 	edge  [loop above] node {a} ()
					edge node {b} (q_3)
			(q_3)	edge [loop below] node {a,b} ();
		\end{tikzpicture}
	}

			\item {\bf [Sipser 1.5 (c)]} (2,0 pt) $\{\omega$ | $\omega$ não contém a subcadeia {\sf ab} nem {\sf ba}$\}$
			
			\vspace*{0.3cm}
			
			{\color{blue}
				\begin{tikzpicture}
				[shorten >=1pt,node distance=2cm,on grid,auto] 
					\node[state,initial,accepting] (q_0)   {$q_0$}; 
					\node[state, accepting] (q_1) [above right=of q_0] {$q_1$}; 
					\node[state, accepting] (q_2) [below right=of q_0] {$q_2$}; 
					\node[state](q_3) [above right=of q_2] {$q_3$};
					\path[->] 
					(q_0) 	edge  node {a} (q_1)
							edge  node  {b} (q_2)
					(q_1)	edge [loop above] node {a} ()
							edge  node  {b} (q_3)
					(q_2)	edge  node  {a} (q_3)
							edge [loop below] node {b} ()
					(q_3)	edge [loop right] node {a,b} ();
				\end{tikzpicture}
			}
				
			\item {\bf [Sipser 1.6 (g)]} (1,0 pt) $\{\omega$ | o comprimento de $\omega$ é no máximo 5$\}$
			
			\vspace*{0.3cm}
			
			{\color{blue}
				\begin{tikzpicture}
					[shorten >=1pt,node distance=2cm,on grid,auto] 
					\node[state,initial,accepting] (q_0)   {$q_0$}; 
					\node[state, accepting] (q_1) [right=of q_0] {$q_1$}; 
					\node[state, accepting] (q_2) [right=of q_1] {$q_2$};
					\node[state, accepting] (q_3) [right=of q_2] {$q_3$};
					\node[state, accepting] (q_4) [below=of q_3] {$q_4$};
					\node[state, accepting] (q_5) [left=of q_4] {$q_5$};
					\node[state] (q_6) [left=of q_5] {$q_6$};
					\path[->] 
					(q_0) 	edge  node {a,b} (q_1)
					(q_1) 	edge  node {a,b} (q_2)
					(q_2) 	edge  node {a,b} (q_3)
					(q_3) 	edge  node {a,b} (q_4)
					(q_4) 	edge  node {a,b} (q_5)
					(q_5) 	edge  node {a,b} (q_6)
					(q_6)	edge [loop left] node {a,b} ();
				\end{tikzpicture}
			}
		\end{enumerate}

\end{enumerate}

\end{document}