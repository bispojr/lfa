\documentclass[12pt,a4paper,oneside]{article}

\usepackage[utf8]{inputenc}
\usepackage[portuguese]{babel}
\usepackage[T1]{fontenc}
\usepackage{amsmath}
\usepackage{amsfonts}
\usepackage{amssymb}
\usepackage{graphicx}
\usepackage{xcolor}
\usepackage{multicol}
\usepackage{tikz}
\usetikzlibrary{automata,positioning}

% Definindo novas cores
\definecolor{verde}{rgb}{0.25,0.5,0.35}

\author{\\Universidade Federal de Goiás (UFG) - Regional  Jataí\\Bacharelado em Ciência da Computação \\Linguagens Formais e Autômatos \\Esdras Lins Bispo Jr.}

\date{27 de novembro de 2017}

\title{\sc \huge Segundo Teste}

\begin{document}

\maketitle

{\bf ORIENTAÇÕES PARA A RESOLUÇÃO}

\small
 
\begin{itemize}
	\item A avaliação é individual, sem consulta;
	\item A pontuação máxima desta avaliação é 10,0 (dez) pontos, sendo uma das 06 (seis) componentes que formarão a média final da disciplina: quatro testes, uma prova e exercícios-bônus;
	\item A média final ($MF$) será calculada assim como se segue
	\begin{eqnarray}
		MF & = & MIN(10, S) \nonumber \\
		S & = & (\sum_{i=1}^{4} 0,2.T_i ) + 0,2.P  + EB\nonumber
	\end{eqnarray}
	em que 
	\begin{itemize}
		\item $S$ é o somatório da pontuação de todas as avaliações,
		\item $T_i$ é a pontuação obtida no teste $i$,
		\item $P$ é a pontuação obtida na prova, e
		\item $EB$ é a pontuação total dos exercícios-bônus.
	\end{itemize}
	\item O conteúdo exigido desta avaliação compreende o seguinte ponto apresentado no Plano de Ensino da disciplina: (2) Autômatos Finitos Determinísticos, e (3) Autômatos Finitos Não-Determinísticos.
\end{itemize}

\begin{center}
	\fbox{\large Nome: \hspace{10cm}}
\end{center}

\newpage

\begin{enumerate}
	
	\section*{Segundo Teste}
	
	\item (5,0 pt) Dê o diagrama de estados dos {\bf AFNs} que reconhecem as seguintes linguagens. Admita em todos os itens que o alfabeto é  $\{0,1\}$.
		\begin{enumerate}
			\item {\bf [Sipser 1.7 (c)]} (1,5 pt) \\$\{\omega$ | $\omega$ contém um número par de {\sf 0}s, ou contém exatamente dois {\sf 1}s$\}$.
			
			\vspace*{0.3cm}
			
			{\color{blue}
				
				\begin{tikzpicture}[->,>=stealth,shorten >=1pt,auto,node distance=1.5cm,
				semithick]
					\node[state,initial] 	(q1)   						{$q_1$}; 
					\node[state, accepting] (q2)	[above right=of q1]	{$q_2$};
					\node[state] 			(q3)	[right=of q2]		{$q_3$};
					\node[state] 			(q4)	[below right=of q1]	{$q_4$};
					\node[state] 			(q5)	[right=of q4]		{$q_5$};
					\node[state, accepting] (q6)	[right=of q5]		{$q_6$};
					\path[->] 
					(q1) edge [bend left]	node {$\epsilon$} (q2)
						 edge [bend right]	node {$\epsilon$} (q4)
					(q2) edge [loop above] 	node {1} ()
						 edge [bend left]	node {0} (q3)
					(q3) edge [loop above] 	node {1} ()
						 edge [bend left]	node {0} (q2)
					(q4) edge [loop above] 	node {0} ()
						 edge [bend left]	node {1} (q5)
					(q5) edge [loop above] 	node {0} ()
						 edge [bend left]	node {1} (q6)
					(q6) edge [loop above] 	node {0} ();
				\end{tikzpicture}
			}
			\item {\bf [Sipser 1.9 (a)]} (2,0 pt) $A \circ B$, em que \\$A = \{\omega$ | o comprimento de $\omega$ é no máximo 5$\}$ e
			\\$B = \{\omega$ | toda posição ímpar de $\omega$ é um {\sf 1}$\}$.
			
			\vspace*{0.3cm}
			
			{\color{blue}
				
				\begin{tikzpicture}[->,>=stealth,shorten >=1pt,auto,node distance=1.5cm,
				semithick]
					\node[state,initial] 	(q0)   						{$q_0$}; 
					\node[state] 			(q1)   	[above=of q0]		{$q_1$};
					\node[state] 			(q2)	[right=of q1]		{$q_2$};
					\node[state] 			(q3)	[right=of q2]		{$q_3$};
					\node[state] 			(q4)	[right=of q3]		{$q_4$};
					\node[state] 			(q5)	[right=of q4]		{$q_5$};
					\node[state] 			(q6)	[below=of q3]		{$q_6$};
					\node[state, accepting] (q7)	[below=of q6]		{$q_7$};
					\path[->] 
					(q0) 	edge [bend left]	node {0,1} (q1)
							edge [bend right]	node {$\epsilon$} (q6)
					(q1) 	edge [bend left]	node {0,1} (q2)
							edge [bend right]	node {$\epsilon$} (q6)
					(q2) 	edge [bend left]	node {0,1} (q3)
							edge [bend right]	node {$\epsilon$} (q6)
					(q3) 	edge [bend left]	node {0,1} (q4)
							edge [bend right]	node {$\epsilon$} (q6)
					(q4) 	edge [bend left]	node {0,1} (q5)
							edge [bend left]	node {$\epsilon$} (q6)
					(q5) 	edge [bend left]	node {$\epsilon$} (q6)
					(q6) 	edge [bend left]	node {1} (q7)
					(q7) 	edge [bend left]	node {0,1} (q6);
				\end{tikzpicture}
			}
			\item {\bf [Sipser 1.10 (b)]} (1,5 pt) $A^*$, em que \\$A = \{\omega$ | $\omega$ contém ao menos dois {\sf 0}s e no máximo um {\sf 1}s$\}$.
			
			\vspace*{0.3cm}
			
			{\color{blue}
				
				\begin{tikzpicture}[->,>=stealth,shorten >=1pt,auto,node distance=1.5cm,
				semithick]
					\node[state,initial, accepting]	(q0)   				{$q_0$}; 
					\node[state] 			(q1)   	[right=of q0]		{$q_1$};
					\node[state] 			(q2)	[below=of q1]		{$q_2$};
					\node[state, accepting]	(q3)	[right=of q2] {$q_3$};					
					\node[state] 			(q4)	[below left=of q2]		{$q_4$};
					\node[state] 			(q5)	[right=of q4]		{$q_5$};
					\node[state, accepting]	(q6)	[below right=of q3]		{$q_6$};
					\path[->] 
					(q0) 	edge [bend left]	node {$\epsilon$} (q1)
					(q1) 	edge [bend right]	node {0} (q2)
							edge [bend right]	node {1} (q4)
					(q2) 	edge [bend left]	node {0} (q3)
							edge [bend right]	node {1} (q5)
					(q3) 	edge [loop above]	node {0} ()
							edge [bend left]	node {1} (q6)
							edge [bend right]	node {$\epsilon$} (q1)
					(q4) 	edge [bend right]	node {0} (q5)
					(q5) 	edge [bend right]	node {0} (q6)
					(q6) 	edge [bend left]	node {$\epsilon$} (q1)
							edge [loop right]	node {0} ();
				\end{tikzpicture}
			}
		\end{enumerate}
	
	\item (5,0 pt) {\bf [Sipser 1.31]} Para qualquer cadeia $\omega = \omega_1 \omega_2 \ldots \omega_n$, o reverso de $\omega$, chamado de $\omega^{\mathcal{R}}$, é a cadeia $\omega$ em ordem reversa, $\omega_n \ldots \omega_2 \omega_1$. Para qualquer linguagem $A$, faça que $A^{\mathcal{R}} = \{ \omega^{\mathcal{R}}$ | $\omega \in A\}$. Mostre que se $A$ é regular, então $A^{\mathcal{R}}$ também é regular.
	
	\vspace*{0.3cm}
	
	{\color{blue}
		{\bf Prova:} Se $A$ é regular, então existe um AFD $M_A = (Q_A, \Sigma_A, \delta_A, q_A, F_A)$ que a reconhece. Iremos construir o AFN $M = (Q, \Sigma, \delta, q_0, F)$, a partir de $M_A$, que reconhece $A^\mathcal{R}$. Apresentamos os elementos de $M$ a seguir:
			\begin{itemize}
				\item $Q = Q_A \cup \{q_0\}$;
				\item $\Sigma = \Sigma_A$;
				\item $\delta(q,a) = \left\{\begin{array}{rl}
				\{r\}, 		& \text{se } \delta_A(r,a) = q\\
				F_A, 			& \text{se } q=q_0 \text{ e } a=\epsilon\\
				\emptyset, 			& \text{caso contrário.}\\
				\end{array} \right.$\\
				em que $q \in Q$, $r \in Q_A$ e $a \in \Sigma$;
				\item $q_0$ é o estado inicial;
				\item $F = \{q_0\}$.
			\end{itemize}
		Como foi possível construir $M$, logo $A^\mathcal{R}$ é regular $\blacksquare$
	}

\end{enumerate}

\end{document}